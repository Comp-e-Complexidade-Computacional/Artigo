\documentclass[12pt]{article}

\usepackage{etoolbox}
\usepackage{sbc-template}
\usepackage{graphicx,url}
\usepackage[utf8]{inputenc}
\usepackage[brazil]{babel}
\usepackage[none]{hyphenat}

\makeatletter
\patchcmd{\@startsection}
{\@afterindentfalse}
{\@afterindenttrue}
{}{}
\makeatother
     
\sloppy

\title{Análise de Problemas de Localização de Instalações Em de Cenários Contextualizados}

\author{Anthony França, Antônio Neto, Enzo Santana, Franck Vasconcelos,\\
Murilo Mota, Rafael Gonçalves, Rene Marinho}

\address{Universidade Tiradentes (UNIT)\\}
\date{2025}

\begin{document} 

\maketitle
     
\begin{resumo}
O presente trabalho visa apresentar um panorama das diferentes definições dos Problemas de Localização de Instalações (Facility Location Problems -- FLPs), por meio da construção de cenários ilustrativos que exemplificam suas distintas aplicações e contextos de manifestação. Ademais, busca-se destacar as metodologias desenvolvidas para tratar cada variante do problema, ressaltando como diferentes abordagens podem ser aplicadas conforme a natureza específica de cada cenário.
\end{resumo}

\section{Introdução}

A busca por um posicionamento ideal de instalações é central para a provisão de serviços, sejam eles comerciais ou públicos, e para a integração regional por meio de infraestruturas físicas, como pontes, aeroportos e terminais rodoviários. No campo da Pesquisa Operacional, esse desafio é formulado como o Problema de Localização de Instalações (PLI), cujo objetivo é selecionar locais de instalação que maximizem o atendimento à demanda e a acessibilidade dos usuários. Essas decisões envolvem diferentes critérios, tais como maximizar a cobertura da demanda, minimizar os custos operacionais e assegurar um atendimento mínimo a determinados grupos de clientes (SOUSA FILHO, 2012; RUSHTON, 1979).

Várias formulações matemáticas têm sido propostas para tratar esses objetivos, destacando-se, entre elas, o Problema de Localização de Máxima Cobertura (PLMC) e o Problema das P-Medianas. Tais modelos dependem de fatores como a distribuição geográfica da população e os tempos de deslocamento para determinar soluções viáveis e eficientes. De fato, estudos recentes confirmam que esses critérios de localização buscam garantir a acessibilidade, ponderando simultaneamente a demanda por serviços e as distâncias de viagem (KUO e KUNG, 2025).

Entre os cenários em que os PLI se manifestam, destaca-se a relação entre instalações e clientes. Nos casos em que a instalação a ser aberta é escolhida dentre um conjunto de locais potenciais e os clientes são alocados a essas instalações visando minimizar o custo de atendimento da demanda, tem-se a formulação conhecida como Problema de Localização de Instalações sem Limitação de Capacidade (Uncapacitated Facility Location Problem -- UFLP). Por outro lado, quando as demandas dos clientes devem ser atendidas exclusivamente por instalações com capacidade limitada, sem possibilidade de fracionar a demanda entre diferentes locais, o problema é classificado como Problema de Localização de Instalações Capacitado de Fonte Única (Single-Source Capacitated Facility Location Problem -- CFLP) (BÜSING et al., 2025).

Tanto o UFLP quanto o CFLP baseiam-se em uma lógica de minimização de custos de transporte entre clientes e instalações, desconsiderando fatores subjetivos, como a preferência dos usuários por determinada instalação (o que pode influenciar sua propensão a utilizá-la). Nessas formulações, o problema assume caráter de NP-difícil, pois é necessário considerar diversas combinações de alocação de clientes a instalações (KANG et al., 2023).

Grande parte das formulações propostas adota cenários determinísticos, em que as variáveis de demanda e tempo são tratadas como valores fixos. Contudo, dada a natureza NP-difícil dos problemas de otimização, há limitações inerentes na busca por soluções ótimas. Mesmo os modelos clássicos, que fornecem descrições generalistas adequadas, apresentam restrições significativas em contextos com alta incerteza ou dinamicidade, que refletiriam melhor a realidade. Assim, a literatura dedica atenção à incorporação de incertezas e aspectos dinâmicos nos modelos de localização, por meio, por exemplo, de programação estocástica e análise de cenários (OWEN e DASKIN, 1998).

Nas próximas seções, serão apresentadas considerações sobre a motivação e os objetivos deste trabalho. Em seguida, traçaremos um panorama dos principais estudos na área de localização de instalações, com ênfase nos trabalhos já citados (especialmente na metodologia de análise proposta por Rushton e na revisão de Owen e Daskin). Posteriormente, na seção de Metodologia, descreveremos como o tema foi abordado nos cenários analisados, e por fim serão apresentados os resultados obtidos.

\subsection{Importância do Tema}

Apesar das limitações em simular todos os cenários possíveis em problemas de localização de instalações, existe um amplo conjunto de pesquisas nessa área, no qual diversas abordagens foram propostas para casos extremos. Isso possibilita utilizar o conhecimento atual do campo para realizar análises que vão desde a infraestrutura existente até predições de desempenho futuro em determinada localidade. Nesse contexto, aplicar metodologias de FLP, mesmo que de forma simplificada, pode ajudar a refletir criticamente sobre como o espaço geográfico tem sido utilizado ao longo do tempo e como poderá ser utilizado no futuro (RUSHTON, 1979).

\subsection{Objetivos}

Este trabalho pretende realizar análises baseadas em metodologias de localização de instalações (FLP), adaptadas à realidade dos cenários sob estudo. Por meio dessas análises, busca-se evidenciar diferentes perspectivas ao confrontar cada cenário com as possibilidades que ele apresenta. Assim, adotamos um paradigma analítico restrito à realidade dos dados disponíveis, procurando extrair percepções a partir dos resultados obtidos.

\section{Trabalhos Relacionados}

Segundo Owen e Daskin (1998), a teoria da localização de instalações iniciou-se em 1909 com Alfred Weber, que buscou posicionar um depósito de modo a minimizar a distância total aos clientes. Esses trabalhos iniciais foram essencialmente analíticos e restritos a formulações matemáticas e representações gráficas básicas, com aplicação prática muito limitada. Somente na década de 1960, com o advento da computação, pesquisadores puderam propor métodos mais aplicáveis a cenários reais de localização. Mesmo assim, as formulações clássicas mantiveram-se estáticas e determinísticas, assumindo parâmetros de demanda e tempos de deslocamento fixos, o que restringe sua flexibilidade diante de condições variáveis.

Estudos mais recentes têm buscado expandir essas abordagens tradicionais incorporando modularidade e dinamicidade. Alarcon-Gerbier e Buscher (2022) ressaltam que a capacidade modular (unidades de produção realocáveis ou expansíveis) e as instalações móveis vêm recebendo atenção crescente, com aplicações emergentes em diversos setores. De modo similar, Kang et al. (2023) propuseram um modelo de localização de serviços que leva em conta a preferência dos clientes pelas instalações, além de considerar explicitamente as capacidades das unidades, evidenciando a complexidade adicional introduzida por esse fator. Nesses modelos modernos, devem ser ponderados não apenas custos de transporte e cobertura de demanda, mas também escolhas comportamentais dos usuários, ampliando os requisitos de viabilidade do sistema. Nesse contexto, trabalhos recentes destacam a alta complexidade desses problemas, por exemplo, Büsing et al. (2024) mostram que determinar uma alocação viável no problema capacitado de fonte única com preferências (SSCFLP-CP) já é \textit{NP}-completo, ilustrando o caráter intratável da variante.

As proposições iniciais forneceram uma base conceitual sólida, mas eram restritas ao funcionamento ideal das instalações e a atendimentos simplificados. Com o avanço da área, passaram a ser considerados cenários mais complexos, como interrupções no sistema. Um exemplo notável é o estudo de Ramshani et al. (2019), que abordam um modelo de localização de dois níveis (\textit{Two-Level UFLP}) sob incertezas de disrupção. Os autores desenvolvem formulações matemáticas e heurísticas (por exemplo, busca tabu) para lidar com paradas de funcionamento em pontos da rede, mostrando como essas descontinuações afetam a seleção de locais e alocações. Esses avanços ilustram a tendência atual de modelar problemas de localização de forma mais realista, incorporando aspectos de resiliência operacional e logística associada às instalações.

\section{Método Explicado}

Nos primeiros trabalhos acerca do problema de localizações, desenvolvidos por Alfred Weber, foi proposto o primeiro modelo das atuais definições de P-Mediano, P-Minimax e P-

\subsection{Posicionamento de Farmácias}

No seguinte cenário analisamos a aplicabilidade de um modelo de OFLP para identificar as disposições mais adequadas ao posicionamento de farmácias, com foco no bairro Centro, em Aracaju, considerando aspectos geográficos, fluxo de pessoas e possibilidade de estabelecimento. Entre os fatores analisados a proximidade de áreas comerciais e distribuidoras de medicamentos, bem como o acesso à infraestrutura necessária para a operação do estabelecimento.

Com o intuito de avaliar a versatilidade do modelo, trabalhamos em cima de um primeiro cenário no qual supomos o estabelecimento de uma farmácia sem bandeira, 

Com o intuito de avaliar a versatilidade do modelo, são construídos diferentes cenários que refletem a realidade da região selecionada, permitindo também inferir sobre sua aplicabilidade em outras localidades. Em alguns cenários, desconsideram-se as farmácias já existentes, de modo a identificar potenciais novas geolocalizações estratégicas, independentemente da concorrência.

Para aumentar a precisão dos resultados, a análise é restrita às áreas com maior intensidade de atividade comercial, uma vez que estas oferecem condições mais adequadas para a coleta de dados. A abordagem combina critérios de acessibilidade, visibilidade e conveniência para os clientes, com o objetivo de maximizar a eficiência operacional e a cobertura de mercado. Os resultados obtidos fornecem subsídios práticos tanto para o planejamento urbano quanto para estratégias de expansão do setor farmacêutico no bairro.

\section{Resultados}

Section titles must be in boldface, 13pt, flush left. There should be an extra
12 pt of space before each title. Section numbering is optional. The first
paragraph of each section should not be indented, while the first lines of
subsequent paragraphs should be indented by 1.27 cm.

\subsection{Subsections}

The subsection titles must be in boldface, 12pt, flush left.

\section{Conclusões}\label{sec:figs}


\begin{thebibliography}{99}

\bibitem{SousaFilho2012} FILHO, G. S. et al. Uma arquitetura e ferramentas para problemas de localização de facilidades no setor público. São Paulo: SBC, 2012. Disponível em: <https://sol.sbc.org.br/index.php/sbsi/article/view/14428>

\bibitem{Rushton1979} RUSHTON, G. Optimal Location of Facilities. [s.l.] Department of Geography, University of Iowa, jan. 1979.

\bibitem{KuoKung2025} KUNG, L.-C.; CHUANG, J.-S.; KUO, Y.-T. Optimal Allocation of Capacitated Facilities considering Time-dependent User Preference for User Number Maximization. Disponível em: <https://ssrn.com/abstract=4276750>. Acesso em: 5 set. 2025. 

\bibitem{Buesing2025} BÜSING, C.; GERSING, T.; WREDE, S. Insights into the computational complexity of the single-source capacitated facility location problem with customer preferences. [s.l.] Optimization Online, dez. 2024. 

\bibitem{Kang2023} KANG, C.-N. et al. A service facility location problem considering customer preference and facility capacity. Computers \& Industrial Engineering, v. 177, p. 109070, 2023.

\bibitem{OwenDaskin1998} OWEN, S. H.; DASKIN, M. S. Strategic facility location: A review. European Journal of Operational Research, v. 111, p. 423–447, 1998. 

\bibitem{Alarcon-GerbierBuscher2022} ALARCONGERBIER, E.; BUSCHER, U. Modular and mobile facility location problems: A systematic review. Computers \& Industrial Engineering, v. 173, p. 108734, 2022. 

\end{thebibliography}

\end{document}